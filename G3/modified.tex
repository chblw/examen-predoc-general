\chapter{Estimation pour données modifiées}

\begin{definition}{Définitions de modifications}{}
	\begin{itemize}
		\item Une observation est tronquée du haut (tronquée à gauche) à $d$ si lorsque la valeur est inférieure à $d$, elle n'est pas enregistrée mais qu'elle est enregistrée si sa valeur est supérieure à $d$. 
		\item Une observations est tronquée du bas (tronquée à droite) à $u$ si lorsque la valeur est supérieure à $u$, elle n'est pas enregistrée mais qu'elle est enregistrée si sa valeur est inférieure à $u$. 
		\item Une observation est censurée du bas (censurée à gauche) à $d$ si lorsque la valeur est inférieure à $d$, elle est enregistrée comme $d$, mais qu'elle est enregistrée à sa valeur si elle est supérieure à $d$. 
		\item Une observation est censurée du haut (censurée à droite) à $u$ si lorsque sa valeur est supérieure à $u$, elle est enregistrée comme $u$, mais qu'elle est enregistrée à sa valeur si elle est inférieure à $u$. 
	\end{itemize}
\end{definition}

En actuariat, on a des données tronquées à gauche et des données censurées à droite. 

Soit l'ensemble de risque $r_j$, le nombre de personnes observées vivantes à l'age $y_j$. En IARD, $r_j$ est le nombre de polices où la perte observée est supérieure à $y_j$ moins ceux qui ont un déductible plus grand ou égal à $y_j$. 

$$S_n = \left\{  \begin{array}{ll}
	                                1,                                 &           0\leq t < y_1            \\ 
	     \prod_{i = 1}^{j-1} \left(\frac{r_i - s_i}{r_i}\right),       & y_{j-1}\leq t < y_j,~j=2, \dots, k \\ 
	\prod_{i = 1}^{k}\left(\frac{r_i - s_i}{r_i}\right) \text{ ou } 0, &             t\geq y_k              \\ 
\end{array}       \right.$$












