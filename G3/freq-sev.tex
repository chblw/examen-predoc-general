\chapter{Fréquence et sévérité avec modifications de la couverture}

\begin{itemize}
	\item Par perte (\textit{Per-loss}) : $Y^L$
	\item Par paiement (\textit{Per-payment}) : $Y^P$
	\item $Y^P = Y^L \vert Y^L > 0.$
	\item En général, diviser $Y^L$ par $S_X(d)$ pour obtenir $Y^P$
\end{itemize}

\section{Déductibles}

\begin{definition}{Déductible ordinaire}{}
	Un déductible ordinaire transforme la variable en excès-moyen ou en variable translatée. On a 
	$$Y^P = \begin{cases}
	\text{indéfini},& X\leq d,\\
	X - d,& X > d
	\end{cases}$$
	$$Y^L = \begin{cases}
	0,& X\leq d,\\
	X - d,& X > d
	\end{cases}$$
\end{definition}

Relations : 

\begin{center}
	\begin{tabular}{ccc}
		\hline
		&                              $Y^P$                               &                  $Y^L$                  \\ \hline
		densité   &          $\displaystyle \frac{f_{X}(y + d)}{S_{X}(d)}$           &      $\displaystyle f_{X}(y + d)$       \\
		survie    &          $\displaystyle \frac{S_{X}(y + d)}{S_{X}(d)}$           &      $\displaystyle S_{X}(y + d)$       \\
		répartition &     $\displaystyle \frac{F_{X}(y + d) - F_{X}(d)}{S_{X}(d)}$     & $\displaystyle F_{X}(y + d) - F_{X}(d)$ \\
		hasard    & $\displaystyle \frac{f_{X}(y + d)}{S_{X}(y + d)} = h_{X(y + d)}$ &      indéfinie à 0, donc indéfinie      \\
		moyenne   &        $\displaystyle \frac{E[X] - E[X \wedge d]}{S(d)}$         &  $\displaystyle E[X] - E[X \wedge d]$   \\ \hline
	\end{tabular}
\end{center}

\begin{definition}{Déductible franchise}{}
	Un déductible franchise paie le montant au complet si la franchise est atteinte. On a 
	$$Y^P = \begin{cases}
		\text{indéfini}, & X\leq d, \\
		X,               & X > d
	\end{cases}$$
	$$Y^L = \begin{cases}
		0, & X\leq d, \\
		X, & X > d
	\end{cases}$$
\end{definition}


\begin{center}
	\begin{tabular}{ccc}
		\hline
		            &                     $Y^P$                      &      $Y^L$       \\ \hline
		  densité   & $\displaystyle \frac{f_{X}(y)}{S_{X}(d)}, y>d$ & $\displaystyle \begin{cases} F_{X}(d),& y = 0\\ f_{X}(y),& y > d\\ \end{cases}$ \\
		  survie    & $\displaystyle \begin{cases}	1,&0\leq y\leq d\\ \frac{S_X(y)}{S_{X}(d)},& y>d \end{cases}$ & $\displaystyle \begin{cases} S_{X}(d),& 0\leq y\leq d\\ S_{X}(y),& y > d\\ \end{cases}$ \\
		répartition & $\displaystyle \begin{cases}	0,&0\leq y\leq d\\ \frac{F_{X}(y) - F_{X}(d)}{S_{X}(d)},& y>d \end{cases}$ & $\displaystyle \begin{cases} F_{X}(d),& 0\leq y\leq d\\ F_{X}(y),& y > d\\ \end{cases}$ \\
		  hasard    & $\displaystyle \begin{cases}	0,&0< y< d\\ h_{X}(y),& y>d \end{cases}$ & $\displaystyle \begin{cases}	0,&0< y< d\\ h_{X}(y),& y>d \end{cases}$ \\
		  moyenne   & $\displaystyle \frac{E[X] - E\left[X\wedge d\right]}{S(d)} + d$ & $\displaystyle E[X] - E[X \wedge d] + d\left[S(d)\right]$ \\ \hline
	\end{tabular}
\end{center}

\section{Ratio d'élimination de perte et effet de l'inflation sur les déductibles ordinaires}

\begin{definition}{Ratio d'élimination de pertes}{}
	Le ratio d'élimination de pertes est le ratio de la décroissance en paiement espéré avec un déductible ordinaire versus un paiement espéré sans déductible : 
	$$\frac{E[X\wedge d]}{E[X]}$$
\end{definition}

\begin{theoreme}{}{}
	Pour un déductible ordinaire $d$ après inflation uniforme de $1 + r$, l'espérance du coût par perte est
	$$(1 + r) \left\{E[X] - E\left[X \wedge \frac{d}{1 + r}\right]\right\}$$
	\tcblower
	Si $F\left(\frac{d}{1+r}\right) < 1$, l'espérance du coût par paiement est 
	$$\frac{(1 + r) \left\{E[X] - E\left[X \wedge \frac{d}{1 + r}\right]\right\}}{S\left(\frac{d}{1 + r}\right)}.$$
	Savoir le prouver. 
\end{theoreme}

\section{Limites de polices}

\begin{definition}{}{}
	Une police avec limite $u$ paie la perte complète si la perte est inférieure à $u$, et $u$ si la perte est supérieure à $u$. On a 
	$$Y = \begin{cases}
	Y,& y<u\\
	u,& y\geq u.
	\end{cases}$$
\end{definition}

Quelques résultats : 

\begin{itemize}
	\item $\displaystyle F_{Y}(y) = \begin{cases}
	F_{X}(y),& y < u\\
	1,& y\geq u.
	\end{cases}$
	\item $\displaystyle F_{Y}(y) = \begin{cases}
	f_{X}(y),& y < u\\
	1-F_{X}(u),& y = u.
	\end{cases}$
	\item Pour une limite de police $u$, après inflation uniforme $1 + r$, le coût espéré est
	$$(1 + r)E\left[X \wedge \frac{u}{1 + r}\right]$$
\end{itemize}

\section{Coassurance, déductibles et limites}

Dans le cas où la compagnie paie une portion $\alpha$ de la perte, la variable aléatoire est $Y = \alpha X$. La variable aléatoire qui incorpore les quatre modifications du chapitre est 

$$Y^L = \begin{cases}
0,& X< \frac{d}{1 + r}\\
\alpha \left[(1 + r)X - d\right],& \frac{d}{1 + r} \leq X < \frac{u}{1 + r}\\
\alpha(u-d), & X\geq \frac{u}{1 + r}
\end{cases}$$

On note que les quantités sont appliqués dans un ordre particulier : la coassurance est appliquée en dernier. 

\begin{theoreme}{Quelques moments pour les modifications}{}
Le premier moment par perte est
$$E\left[Y^L\right] = \alpha(1 + r)\left\{ E\left[X \wedge \frac{u}{1 + r}\right] - E\left[X \wedge \frac{d}{1 + r}\right]\right\}$$
et le premier moment par paiement est
$$E\left[Y^P\right] = \frac{E\left[Y^L\right]}{1 - F_{X}\left(\frac{d}{1 + r}\right)}.$$
\tcblower
Le deuxième moment par perte est
$$E\left[\left(Y^L\right)^2\right] = \alpha^2 (1 + r)^2 \left\{ E\left[(X \wedge u^*)^2\right]-E\left[(X \wedge d^*)^2\right] -2d^*E\left[X \wedge u^*\right] + 2d^*E\left[X \wedge d^*\right]\right\},$$
où $$u^* = \frac{u}{1 + r} \quad \text{et}\quad d^* = \frac{d}{1 + r}.$$
Pour par-paiement, on a 
$$E\left[\left(Y^L\right)^2\right] = \frac{E\left[\left(Y^L\right)^2\right]}{1 - F_{X}\left(d^*\right)}.$$
Preuve : facile, manipuler les mins et max et prendre l'espérance. À connaître. 
\end{theoreme}

Méthode générale pour calculer les moments : 

$$E[g(x)] = \int_{R} g(x) f(x) dx = \int_{R}g'(x)S(x) dx.$$

La dérivée simplifie souvent l'intégrale. 





