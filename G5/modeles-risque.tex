\chapter{Modélisation des risques}

\section{Modèle stochastique du risque}

\begin{definition}{Modèle stochastique du risque}{}
	On définit la v.a. $X$ par une somme aléatoire 
	\begin{equation}\label{eq:collective}
		X = \begin{cases}
		\sum_{k = 1}^{M}B_k, & M>0\\
		0, & M = 0.
		\end{cases}
	\end{equation}
	
	\begin{itemize}
		\item La v.a. discrète $M$ représente le nombre de sinistres (fréquence)
		\item La v.a. continue et positive $B_k$ correspond au montant du $k^{\text{ième}}$ sinistre (sévérité)
		\item On dit que la v.a. est composée
	\end{itemize}
\end{definition}

\begin{definition}{Modèle collectif vs modèle individuel}{}
	Un modèle collectif du risque est un modèle de la forme \ref{eq:collective} où 
	\begin{itemize}
		\item conditionnel à $N = n$, les v.a. $B_k$ sont i.i.d ;
		\item conditionnel à $N = n$, la suite de v.a. $\{B_k, k = 1, \dots, n\}$ sont i.i.d et ne dépend pas de $n$ ;
		\item la distribution de $N$ ne dépend pas des valeurs $B_k, k = 1, \dots, n$.
	\end{itemize}
	\tcblower
	Un modèle individuel du risque représente la perte agrégée sous la forme d'une somme de $n$ contrats d'assurance : $$S = \sum_{i = 1}^{n}X_i.$$
	\begin{itemize}
		\item Les pertes $X_i$ sont assumés indépendants mais pas nécessairement identiquement distribués ;
		\item La distribution des $X_i$ a habituellement une masse à 0, représentant la probabilité qu'aucun paiement n'est fait pour un contrat. 
	\end{itemize}
\end{definition}


L'espérance de $X$ dans un modèle collectif est 
$$E[X] = E[E[X \vert M]] = E[M \times E[B]] = E[M]E[B].$$
La variance de $X$ est 
$$Var(X) = E_M[Var(X\vert M)] + Var_M(E[X \vert M]).$$
Avec $Var(X \vert M) = M Var(B)$, on obtient
$$Var(X) = E[M]Var(B) + Var(M)E[B]^2.$$

La fonction de répartition est 
\begin{align*}
	F_X(x) & = E[F_X(x) \vert M]                                                      \\
	       & = \Pr(M = 0) + \sum_{k = 1}^{\infty} \Pr(M = k)F_{B_1 + \dots + B_k}(x).
\end{align*}
Pour le modèle collectif, on a 
$$F_X(x) = \sum_{k = 0}^{\infty} p_k F_B^{*k}(x),$$
où $F_B^{*k}(x)$ est le produit de convolution. En dérivant, on obtient
$$f_X(x) = \sum_{k = 0}^{\infty} p_k f_B^{*k}(x).$$

La fgp est 

$$E\left[t^{X}\right] = E_M\left[E\left[t^{X}\vert M\right]\right] = E_M\left[t^{B_1 + \dots + B_k}\right] = E_M\left[\prod_{i = 1}^{k}t^{B_i}\right] \stackrel{iid}{=} E_M\left[P_B(t)^M\right] = P_M(P_B(t)).$$

La fgm est 
$$E\left[e^{tX}\right] = E_M\left[E\left[e^{tX}\vert M\right]\right] = E_M\left[e^{t(B_1 + \dots + B_k)}\right] = E_M\left[\prod_{i = 1}^{k}e^{B_it}\right] \stackrel{iid}{=} E_M\left[M_B(t)^M\right] = P_M(M_B(t)).$$

La transformée de Laplace est 

$$L_X(t) = P_M(L_B(t)).$$

\subsection{Fréquence et sévérité avec modification de la couverture}

Soit $v = \Pr(Y^L = 0)$. On a 
\begin{itemize}
	\item Relation entre montant des pertes par perte et par paiement : $\displaystyle M_{Y^L}(t) = (1-v) + vM_{Y^P}(t)$
	\item Relation entre nombres de pertes par perte et par paiement : $\displaystyle P_{N^P}(t) = P_{N^L}(1-v + vt)$
\end{itemize}
Dans le modèle collectif du risque, on a 
\begin{itemize}
	\item $\displaystyle M_S(t) = P_{N^L}\left(M_{Y^L}(t)\right)$
	\item $\displaystyle M_S(t) = P_{N^P}\left(M_{Y^P}(t)\right)$
\end{itemize}

\section{Approche indemnitaire et forfaitaire}

Un cas particulier du modèle stochastique est où $M$ obéit à une loi Bernouilli. On a 
$$X = M \times B = \begin{cases}
B, & M = 1\\
0, & M = 0.
\end{cases}$$

\begin{definition}{Approche indemnitaire}{}
	Dans l'approche indemnitaire, on suppose qu'un seul sinistre peut arriver dans une période. Les coûts de l'évènement sont définis par la v.a. $B$. On a 
	
	\begin{itemize}
		\item $\displaystyle F_X(x) = 1-q + qF_B(x)$
		\item $\displaystyle E[X] = qE[B]$
		\item $\displaystyle Var(X) = qVar(B) + q(1-q)E[B]^2$
		\item $\displaystyle VaR_\kappa(X) = \begin{cases}
		0, & 0<\kappa<1-q\\
		VaR_{\frac{\kappa - (1-q)}{q}}(B), & 1-q<\kappa<1
		\end{cases}$
		\item $\displaystyle TVaR_\kappa(X) = \frac{qE\left[B\times \mathbbm{1}_{\{X > VaR_\kappa(X)\}}\right] + VaR_\kappa(X)(F_X(VaR_\kappa(X)) - \kappa)}{1 - \kappa}$
	\end{itemize}
\end{definition}

L'approche indemnitaire est utile pour l'assurance vie. Soit $S = X_1 + X_2 + \dots + X_n$. Soit $q_j$, la probabilité de décès de l'individu $j$ et $b_j$, la prestation de décès. Alors, 
\begin{itemize}
	\item $\displaystyle E[S] = \sum_{j = 1}^{n}b_j q_j$
	\item $\displaystyle Var(S) = \sum_{j = 1}^{n}b_j^2 q_j(1-q_j)$
	\item $\displaystyle P_S(t) = \prod_{j = 1}^{n}(1 - q_j + q_j z ^ b_j)$
	\item $\displaystyle M_S(t) = \prod_{j = 1}^{n}(1 - q_j + q_j M_{B_j}(t))$
\end{itemize}

\section{Généralisations des principales lois de fréquence}

\begin{definition}{Loi Poisson-mélange}{}
	Soit une v.a. $\Theta$ de telle sorte que $E[\Theta] = 1, Var(\Theta) < \infty$ et $M_\Theta(x)$ existe. La distribution de fréquence est définie selon
	$$(M\vert \Theta = \theta) \sim Pois(\lambda \theta).$$
	On a 
	\begin{itemize}
		\item $\displaystyle E[M] = E_\Theta[E[M \vert \Theta]]$
		\item $\displaystyle Var(X) = \lambda + \lambda^2 Var(\Theta)$
		\item $\displaystyle P_M(t) = M_\Theta(\lambda(t-1))$
		\item $\displaystyle \Pr(M = k) = \int_{0}^{\infty} e^{-\lambda t} \frac{(\lambda \theta)^k}{k!}dF_\theta(\theta)$
	\end{itemize}
\end{definition}

\section{Sommes aléatoires et allocation}

Soit la v.a. $X$ définie selon un modèle stochastique du risque. Si $B$ est continue et positive, on a 
$$TVaR_\kappa(X) = \frac{1}{\kappa}\sum_{k = 1}^{\infty} f_M(k) E\left[ \left(\sum_{j = 1}^{k}B_j\right)\times \mathbbm{1}_{\left\{\sum_{j = 1}^{k}B_j > VaR_\kappa(X)\right\}}\right].$$
On présente deux règles d'allocation du capital. 

\begin{definition}{Règle basée sur la variance}{}
	Les parts alloués aux v.a. $B$ et $M$ sont
	\begin{align*}
	C_\kappa^{Var}(M) &= \frac{Var(E[X\vert M])}{Var(X)} TVaR_\kappa(X) = \frac{Var(M)E[B]^2}{Var(X)} TVaR_\kappa(X)\\
	C_\kappa^{Var}(B) &= \frac{E(E[X\vert M])}{Var(X)} TVaR_\kappa(X) = \frac{E[M]Var(B)}{Var(X)} TVaR_\kappa(X)
	\end{align*}
\end{definition}

\begin{definition}{Régle basée sur la TVaR}{}
	Avec la décomposition 
	$$E[X \times \mathbbm{1}_{\{X>b\}}] = \underbrace{E_M[(X - E[X\vert M])\times \mathbbm{1}_{\{X>b\}}]}_{\text{part allouée à }M} + \underbrace{E_M[E[X\vert M]\times \mathbbm{1}_{\{X>b\}}]}_{\text{part allouée à }B},$$
	On déduit
	$$C_\kappa^{TVaR}(M) = \frac{E_M\left[E[X \vert M] \times \mathbbm{1}_{\{X>VaR_\kappa(X)\}}\right]}{1 - \kappa}$$
	et
	$$C_\kappa^{TVaR}(B) = TVaR_\kappa(X) - C_\kappa^{TVaR}(M).$$
\end{definition}

\section{Pertes financières}

\begin{definition}{}{}
	Soit $V(0)$, la v.a. qui représente le capital initial investi dans un fonds et $V(1)$, la v.a. qui représente la valeur du fonds au temps $t = 1$. La perte liée à cet investissement est défini selon la v.a. $L = V(0) - V(1)$. Le rendement instantanné pour une période est défini selon la v.a. $R$, alors, $L = V(0) - V(0)e^R$. Si la v.a. $R$ est continue, on a 
	$$VaR_\kappa(L) = V(0) + VaR_\kappa(-V(1)) = V(0) - VaR_{1 - \kappa}(V(1))$$
	et
	$$TVaR_\kappa(L) = V(0) - \frac{1}{1 - \kappa} \left(E[X]-\kappa TVaR_{1 - \kappa}(X)\right).$$
\end{definition}