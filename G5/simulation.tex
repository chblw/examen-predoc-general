\chapter{Méthodes de simulation}

\section{Méthode de base}

Un algorithme pour simuler la réalisation d'une v.a. $X^{(j)}, j = 1, 2, \dots, m$ à partir de la fonction de répartition d'une fonction. Pour $j = 1, 2, \dots, m$, 
\begin{enumerate}
	\item simuler la réalisation $U^{(j)} \sim Unif(0, 1)$ à partir d'un GNPA ou \texttt{runif} ;
	\item calculer la simulation $X^{(j)}$ de $X$ avec $X^{(j)} = F_{X}^{-1}\left(U^{(j)}\right)$.
\end{enumerate}

\subsection{Simulations de v.a. définis par un mélange}

Soit une v.a. $\Theta$ et une v.a. $X$ dont la fonction de réparition conditionnelle de $(X \vert \Theta = \theta)$ est $F_{X\vert \Theta = \theta}$ et la fonction quantile est $F_{X \vert \Theta = \theta}^{-1}$. Pour simuler des réalisations de $\underline{X}$, on procède avec les étapes suivantes
\begin{enumerate}
	\item on simule une réalisation $\Theta^{(j)}$ de $\Theta$ ;
	\item on produit une réalisation $X$
\end{enumerate}

\subsection{Simulation  de somme aléatoire}

Soit $X$ une v.a. définie par le modèle stochastique. La procédure pour simuler des réalisations de $X$ est 
\begin{enumerate}
	\item Simuler une réalisation $M^{(j)}$ de la v.a. $M$. 
\end{enumerate}





































