\chapter{Taux d'intérêt}

En général, répète les deux premiers chapitres de la partie \ref{part:math-fin} sur les mathématiques financières. Différents types de taux d'intérêt sont montrés. 

\section{Types de taux}

Les taux d'intérêt sont souvent exprimés en points de base. Un point de base est 0.01\% par année. 

\begin{itemize}
	\item \textbf{Taux du trésor} : Taux d'intérêt obtenu sur des bons du trésor. Ils sont utilisés par des gouvernements pour emprunter de l'argent. 
	\item \textbf{LIBOR} : London Interbank Offered Rate, le taux d'emprunt à court-terme des banques (non sécurisé).
	\item \textbf{Taux Fed Funds} : Les institutions financières aux états unis doivent maintenir un montant liquide avec la federal reserve, dépendant de leurs actifs et obligations. À la fin de la journée, certaines institutions financières ont un surplus / déficit qui doit être réglé au cours de la nuit. Le taux imposé pour cette nuit est le taux federal funds. 
	\item \textbf{Taux repo} (Repo rate) : taux d'emprunt sécurisé (repurchase agreement). Ils vendent et rachètent un actif pour agir comme emprunt à court terme. La différence entre le prix de vente et de rachat est l'intérêt, référé au taux repo.
	\item \textbf{Taux sans risque} : Les produits dérivés dépendent du taux sans risque. On utilise souvent el taux LIBOR comme taux sans risque (même si le taux LIBOR n'est pas sans risque). 
\end{itemize}

\section{Mesure de l'intérêt}

Savoir comment composer $m$ fois par année, continuellement. 

\section{Taux zero coupon}

Le taux d'intérêt d'un zero-coupon $n$ années est le taux d'intérêt accru sur un investissement qui commence aujourd'hui est dure $n$ années. L'intérêt et le principal initial est repayé à la fin des $n$ années, il n'y a aucun paiement intermédiaire. C'est appelé un zero-coupon car c'est un obligation (dette) sans coupon. 


\section{Tarification des obligations}

Un obligation est une dette que l'investisseur achète aujourd'hui. En retour, il reçoit des coupons (paiements intermédiaires). L'investisseur reçoit le paiement initial à la fin de la durée du coupon. Le prix théorique d'un obligation est déterminé par la valeur actualisée de tous les sorties de fonds : les coupons et le paiement à échéance. 

Il peut y avoir un taux d'escompte pour tous les coupons, ou utiliser différents taux zéro-coupon pour actualiser chaque coupon. 

\textbf{Taux de rendement des obligations} : Le taux de rendement de l'obligation est le taux d'escompte à utiliser si tous les maturités ont le même taux d'escompte annuel. On doit souvent résoudre des équations non-linéaires pour obtenir le taux de rendement.

\textbf{Taux valeur nominale} : Le taux de valeur nominale (par yield) pour un obligation est le taux de rendement dont la valeur actualisée des paiements est égal à la valeur nominale.

\section{Détermination des taux zero-coupon des bons du trésor}

Pour déterminer les taux zéro-coupon sur les bons du trésor, il faut décomposer un bon du trésor en bandes (strips). Ces obligations zéro-coupon sont créés synthétiquement et vendu comme coupons individuels, séparé de la valeur nominale de l'obligation. 


Une autre méthode est la méthode \textit{bootstrap}. On prend des bons du trésor avec plusieurs maturités. Les taux zéro-coupon sont déterminés par les coupons avec aucun coupon. Pour les obligations avec des coupons, les coupons sont actualisés avec les taux zéro-coupon déterminés avec maturités plus courtes correspondants. 

\section{Taux forward}

Les taux d'intérêt forward sont les taux d'intérêt implicites des taux zéro-coupon. Si le taux d'intérêt 0 coupon de maturité 1 et 2 sont respectivement 3\% et 4\%, le taux d'intérêt entre l'année 1 et 2 est le taux d'intérêt tel que $(1.03)(1+x) = (1.04)^2$. 

\section{Accords de taux forwards}

Un accord de taux d'intérêt forward est un accord gré-à-gré pour déterminer le taux d'intérêt pour emprunter une certaine valeur nominale pour une période de temps dans le futur. Les taux LIBOR sont souvent utilisés comme taux de référence : les accords paient la différence. 

\textbf{Valuation} : Lorsqu'un contrat à terme est entré, la valeur du contrat est 0 car aucune valeur n'est échangée. Par contre, lorsque le prix change, la valeur du contrat change. La valeur marchande d'un produit dérivé est son évaluation à la valeur de marché (mark-to-market, MTM)

\section{Duration}

La duration d'un obligation mesure la durée de temps qu'un investisseur doit attendre pour recevoir des paiements. La duration d'un obligation zéro-coupon de $n$ années est $n$ années. Un obligation $n$ années avec des coupons a une duration de moins que $n$ années. 

Un obligation retourne des flux monétaires de valeur $c_i$ au temps $t_i, 1\leq i\leq n$. Le prix de l'obligation $B$ et le taux de rendement interne est déterminé par la relation
$$B = \sum_{i = 1}^{n}c_i e^{-yt_i}.$$
La duration de l'obligation est notée par 
$$D = \frac{1}{B} \sum_{i = 1}^{n}t_ic_ie^{-yt_i} = \sum_{i = 1}^{n} t_i \left[\frac{c_i e^{-yt_i}}{B}\right],$$
qui est une moyenne dy temps, pondéré par la valeur des paiements. Pour calculer la duration, toutes les actualisations sont faites avec le taux de rendement de l'obligation. 

On a l'équation différentielle 
$$\frac{\Delta B}{B} = -D\Delta y,$$
qui permet d'approximer le changement de pourcentage du prix de l'obligation avec le changement dans son taux de rendement. 

La duration est faite avec l'hypothèse d'intérêt composé en continu. Si l'intérêt est composé $m$ fois par année, on a 
$$\Delta B = - \frac{BD \Delta y}{1 + y/m}.$$
La variable $D^* = \frac{D}{1 + y/m}$ est noté la duration mofifiée, qui permet d'écrire
$$\Delta B = -BD^* \Delta y,$$
oỳ $y$ est le taux d'intérêt composé $m$ fois par année. 

\subsection{Portefeuille d'obligations}

La duration $D$ d'un portefeuille est défini comme la moyenne pondérée (par le prix des obligations) des durations des obligations individuelles dans le portefeuille. 

\section{Convexité}

La convexité mesure la courbature des fonctions de taux de rendement. Une mesure de convexité est
$$C = \frac{1}{B} \frac{d^2 B}{dy^2} = \frac{1}{B} \sum_{i = 1}^{n} c_i t_i^2 e^{-yt_i}.$$
De l'expansion de Taylor de $\Delta B$, on a 
$$\frac{\Delta B}{B} = -D\Delta y  \frac{1}{2} C (\Delta y)^2.$$
Pour un portefeuille avec une certaine duration, la convexité d'un portefeuille d'obligations tend à être plus grande quand les coupons sont répartis sur une grande période et plus petite lorsque les paiements sont concentrés à une date. 

\section{Théories de la structure à terme des taux d'intérêt}

\begin{itemize}
	\item \textbf{Expectations theory} : les taux d'intérêt à long terme devraient reflecter les taux d'intérêt à court terme. 
	\item \textbf{Market segmentation theory} : il n'y a aucune relation entre les taux d'intérêt à court, moyen et long terme. 
	\item \textbf{Liquidity preference theory} : plus populaire. Les investisseurs préfèrent préserver leur liquidité et investir dans des produits à court terme. Les emprunteurs désirent sécuriser leur taux d'intérêt pour longtemps et préfèrent les produits à long terme. Par exemple, la majorité des dépôts dans les banques sont sur des produits court terme, et la majorité des prets hypothéquaires sont pour 5 ans. Alors, il y a un désaccord entre les actifs et obligations. 
\end{itemize}

