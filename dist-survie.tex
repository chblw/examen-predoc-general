\chapter{Distributions de survie}

\subsection{Notation}

\begin{center}
	\begin{tabular}{cc}
		\hline
		Symbole &                              Description                               \\ \hline
		$(x)$  &                               âge de $x$                               \\
		$[x]$  &                       âge à la sélection de $x$                        \\
		$X$   &                         v.a. de l'âge au décès                         \\
		$T(x)$  &                 v.a. de la vie de $(x)$, aussi $X - x$                 \\
		$K(x)$  &      v.a. de la vie entière de $x$, aussi $\lfloor T(x) \rfloor$       \\
		$S(x)$  & v.a. de la partie fractionnaire de l'âge au décès, aussi $T(x) - K(x)$ \\ \hline
	\end{tabular}
\end{center}

\begin{center}
	\begin{tabular}{ccc}
		\hline
		   Symbole    &                         Description                         &                                               \\ \hline
		   $s(x)$     &             fonction de survie d'un nouveau né              &             $\overline{F}_{X}(x)$             \\
		  $\mu(x)$    &                     force de mortalité                      & $\displaystyle \frac{f_{X}(x)}{1 - F_{X}(x)}$ \\
		 $\mu_x(t)$   &      force de mortalité pour une sélection à l'âge $x$      &                                               \\
		 $\qx[t]{x}$  & probabilité que $(x)$ décède dans les $t$ prochaines années &              $\Pr(T(x) \leq t)$               \\
		 $\px[t]{x}$  &   probabilité que $(x)$ survive les $t$ prochaines années   &                $\Pr(T(x) > t)$                \\
		$\qx[t|u]{x}$ & probabilité que $(x)$ survive $t$ mais décède avant $t + u$ &              $\Pr(t<T(x)<t + u)$              \\
		$\eringx{x}$  &                                                             &                   $E[T(x)]$                   \\
		    $e_x$     &                                                             &                   $E[K(x)]$                   \\ \hline
	\end{tabular}
\end{center}

Autres relations : 

\begin{itemize}
	\item $\Pr(K(x) = k) = \px[k]{x}q_{x+k} = \qx[k|]{x}$
	\item $F_{K(x)}(y) = \sum_{h = 0}^{\lfloor y \rfloor}\qx[h|]{x}$
\end{itemize}


\begin{center}
	\begin{tabular}{@{}l|llll@{}}
		\toprule
		\multicolumn{1}{c}{Outil} &                                                                                                                \multicolumn{4}{c}{Relations}                                                                                                                 \\
		\cmidrule(r){1-1}\cmidrule(r){2-5} & $F_{X}(x)$                                                    & $s(x)$                                                    & $f_{X}(x)$                                                       & $\mu(x)$                                                      \\ \midrule
		$F_{X}(x)$ &                                                               & $\displaystyle 1 - F_{X}(s)$                              & $\displaystyle F_{X}'(x)$                                        & $\displaystyle \frac{F'_{X}(x)}{1 - F_{X}(x)}$                \\
		$s(x)$ & $\displaystyle 1 - s(x)$                                      &                                                           & $\displaystyle -s'(x)$                                           & $\displaystyle -\frac{s'(x)}{s(x)}$                           \\
		$f_{X}(x)$ & $\displaystyle \int_{0}^{x} f_{X}(t) dt$                      & $\displaystyle \int_{x}^{\infty}f_{X}(t) dt$              &                                                                  & $\displaystyle \frac{f_{X}(x)}{\int_{x}^{\infty}f_{X}(t) dt}$ \\
		$\mu(x)$ & $\displaystyle 1 - \exp\left\{-\int_{0}^{x}\mu(t) dt\right\}$ & $\displaystyle \exp\left\{-\int_{0}^{x}\mu(t) dt\right\}$ & $\displaystyle \mu(x) \exp\left\{-\int_{0}^{x}\mu(t) dt\right\}$ &                                                               \\ \bottomrule
	\end{tabular}
\end{center}

\section{Relations avec tables de mortalité}

On peut obtenir la probabilité de décès avec la relation $$\qx[t]{x} = 1 - \frac{s(x + t)}{s(x)}.$$

Soit $$I_j = \begin{cases}
1, & \text{si l'assuré $j$ survie jusqu'à $x$}\\
0, & \text{sinon}
\end{cases}$$

\begin{center}
	\begin{tabular}{ccl}
		\hline
		    Symbole      &                             Description                             & Formule                                \\ \hline
		     $l_0$       &           nombre initial de nouveaux nés dans la cohorte            &                                        \\
		$\mathcal{L}(x)$ &       v.a. du nombre de survivants d'une cohorte à l'age $x$        & $\displaystyle \sum_{j = 1}^{l_0} I_j$ \\
		  $\Dx[n]{x}$    & v.a. du nombre de décès entre les âges $x$ et $x + n$ d'une cohorte &                                        \\
		     $l_x$       &                                                                     & $E[\mathcal{L}(x)]$                    \\
		  $\dx[n]{x}$    &                                                                     & $E[\Dx[n]{x}] = l_x - l_{x + n}$       \\
		     $m_x$       &                        âge centrale au décès                        &                                        \\
		    $\omega$     &            Oméga, l'âge limite d'une table de mortalité             &                                        \\ \hline
	\end{tabular}
\end{center}

De plus, $\mathcal{L}(x)\sim Bin(l_0, s(x))$. 


\section{Groupe de survie déterministe}

\begin{definition}{}{}
	Un groupe de survie déterministe, représenté par une table de mortalité, a les charactéristiques suivantes : 
	\begin{itemize}
		\item le groupe conteint $l_0$ nouveaux nés initialement
		\item les membres sont sujets à taux de mortalité $q_x$ pour chaque année de leur vie
		\item le groupe est fermé, la taille du groupe change seulement si un membre est décédé. 
	\end{itemize}
\end{definition}

\begin{multicols}{2}
	Récursion avec probabilités de décès :
	\begin{align*}
	l_1 &= l_0(1 - q_0) \\
	l_2 &= l_1(1 - q_1) \\
	l_3 &= l_2(1 - q_2) \\
	& \quad\vdots\\
	l_x &= l_{x-1}(1 - q_{x-1}) = l_0(1 - \qx[x]{0}).
	\end{align*}
	\columnbreak
	
	Récursions avec probabilités de survie : 
	\begin{align*}
	l_1 &= l_0 p_0\\
	l_2 &= l_1 p_1 = l_0 p_0p_1\\
	l_3 &= l_2 p_2 = l_0 p_0p_1p_2\\
	&\quad \vdots \\
	l_x &= l_{x-1}p_{x-1} = l_0 \prod_{y = 0}^{x-1}p_y.
	\end{align*}
\end{multicols}

\section{Autres characéristiques}

\begin{multicols}{2}
	\begin{align*}
	\eringx{x} = E[T(x)]&= \int_{0}^{\infty} t \px[t]{x}\mu(x + t)dt\\
	&= \int_{0}^{\infty} \px[t]{x}dt.
	\end{align*}
	\begin{align*}
	E[T(x)^2]&= \int_{0}^{\infty} t^2 \px[t]{x}\mu(x + t)dt\\
	&= \int_{0}^{\infty} 2t\px[t]{x}dt.
	\end{align*}
	\begin{align*}
	e_{x} = E[K(x)]&= \sum_{k = 0}^{\infty} k \px[k]{x}\qx{x + k}\\
	&= \sum_{k = 0}^{\infty}\px[k]{x}.
	\end{align*}
	\begin{align*}
	E[K(x)^2]&= \sum_{k = 0}^{\infty} k \px[k]{x}\qx{x + k}\\
	&= \sum_{k = 0}^{\infty}(2k-1)\px[k]{x}.
	\end{align*}
\end{multicols}

\subsection{Relations de récursion}

\begin{definition}{}{}
	La formule de récursion arrière est sous la forme $$u(x) = c(x) + d(x) u(x + 1)$$\tcblower
	La formule de récursion avant est sous la forme $$u(x + 1) = -\frac{c(x)}{d(x)} + \frac{1}{d(x)}u(x)$$
\end{definition}

\begin{example}{Trouver la relation de récursion arrière pour $\exercicemath{e_x}$. }{}
	On a $$e_x = \sum_{k = 1}^{\infty} \px[k]{x}.$$
	On sort le premier terme de la relation et on obtient
	$$e_x = \px[]{x} + \sum_{k = 2}^{\infty} \px[k]{x}.$$
	On applique un glissement d'indices et on obtient
	$$e_x = \px[]{x} + \sum_{k = 1}^{\infty} \px[k]{x + 1} = \px[]{x} + \px[]{x}e_{x+1}.$$
	On conclut que $u(x) = e_x, c(x) = \px[]{x}$ et $d(x) = p_x$. Le point de départ est $u(\omega) = 0.$
\end{example}

\section{Hypothèses pour les âges fractionnaires}

On utilise l'interpolation linéaire pour les âges fractionnaires, c.-à-d. 
$$s(x + t) = (1-t)s(x) + ts(x + 1).$$
Cette méthode est appelée distribution uniforme des décès (DUD).

\begin{center}
	\begin{tabular}{cc}
		\hline
		Fonction        &         Hypothèse         \\ \hline
		$\qx[t]{x}$       &        $\displaystyle t\qx[]{x}$        \\
		$\mu(x + t)$      &   $\displaystyle \frac{q_x}{1-tq_x}$    \\
		$\qx[1-t]{x+t}$     & $\displaystyle \frac{(1-t)q_x}{1-tq_x}$ \\
		$\qx[y]{x + t}$     &  $\displaystyle \frac{yq_x}{1 - tq_x}$  \\
		$\px[t]{x}$       &        $1 - tq_x$         \\
		$\px[t]{x} \mu(x + t)$ &           $q_x$           \\ \hline
	\end{tabular}
\end{center}

\section{Lois analytique de survie}

\begin{center}
	\begin{tabular}{ccc}
		\hline
		Nom    &      $\mu(x)$       & $s(x)$ \\ \hline
		De Moivre & $(\omega - x)^{-1}$ &   $\displaystyle 1 - \frac{x}{\omega}$     \\
		Gompertz  &       $ Bc^x$       &   $\displaystyle \exp\left\{-\frac{B}{\log c}(c^x - 1)\right\}$     \\
		Makeham  &     $ A + Bc^x$     &   $\displaystyle \exp\left\{-Ax - \frac{B}{\log c}(c^x - 1)\right\}$     \\
		Weibull  &       $kx^n$        &   $\displaystyle \exp\left\{-\frac{k}{n+1}x^{n+1}\right\}$     \\ \hline
	\end{tabular}
\end{center}


\section{Tables de mortalité sélectes et ultimes}

Pour un assuré observé à $(x)$, on pourrait penser avoir plus d'information qu'un assuré observé plus tôt mais qui a survécu jusqu'à $x$. Alors, on se définit une table de mortalité sélecte pour les assurés observés à $(x)$. On note cette sélection dans la notation comme $[x]$. 
\begin{itemize}
	\item Pendant $r$ années (la période de sélection), on a $\qx[]{[x] + 1} \neq \qx[]{x + 1}$. 
	\item Après $r$ années, on a $\qx[]{[x] + r + 1} = \qx[]{x + r + 1}$.
\end{itemize}









